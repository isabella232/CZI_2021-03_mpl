%%%%%%%%%%%%%%%%%%%%%%%%%%%%%%%%%%%%%%%%%
% Medium Length Graduate Curriculum Vitae
% LaTeX Template
% Version 1.1 (9/12/12)
%
% This template has been downloaded from:
% http://www.LaTeXTemplates.com
%
% Original author:
% Rensselaer Polytechnic Institute (http://www.rpi.edu/dept/arc/training/latex/resumes/)
%
% Important note:
% This template requires the res.cls file to be in the same directory as the
% .tex file. The res.cls file provides the resume style used for structuring the
% document.
%
%%%%%%%%%%%%%%%%%%%%%%%%%%%%%%%%%%%%%%%%%

%----------------------------------------------------------------------------------------
%	PACKAGES AND OTHER DOCUMENT CONFIGURATIONS
%----------------------------------------------------------------------------------------

\documentclass[margin]{res} % Use the res.cls style, the font size can be changed to 11pt or 12pt here

%\usepackage{helvet} % Default font is the helvetica postscript font
%\usepackage{newcent} % To change the default font to the new century schoolbook postscript font uncomment this line and comment the one above
\usepackage{hyperref}
\usepackage{textcomp}
\usepackage{ulem}
%\setlength{\textwidth}{5.5in} % Text width of the document
\usepackage{multicol}

\makeatletter
\renewenvironment{thebibliography}[1]
{
%     \begin{multicols}{2}[
\setlength{\leftskip}{-\sectionwidth}

\moveleft.5\sectionwidth\vbox{\hrule width\textwidth}\smallskip % Horizontal line after name; adjust line thickness by changing the '1pt'

%]%
      \list{\@biblabel{\@arabic\c@enumiv}}%
           {\settowidth\labelwidth{\@biblabel{#1}}%
            \leftmargin\labelwidth
            \advance\leftmargin\labelsep
            \@openbib@code
            \usecounter{enumiv}%
            \let\p@enumiv\@empty
            \renewcommand\theenumiv{\@arabic\c@enumiv}}%
      \sloppy
      \clubpenalty4000
      \@clubpenalty \clubpenalty
      \widowpenalty4000%
      \sfcode`\.\@m}
     {\def\@noitemerr
       {\@latex@warning{Empty `thebibliography' environment}}%
      \endlist%\end{multicols}}
}
\makeatother
\begin{document}

%----------------------------------------------------------------------------------------
%	NAME AND ADDRESS SECTION
%----------------------------------------------------------------------------------------

\moveleft\hoffset\vbox{\large\bf Thomas A Caswell} % Your name at the top

\moveleft\hoffset\vbox{\hrule width\resumewidth height 1pt}\smallskip % Horizontal line after name; adjust line thickness by changing the '1pt'

%----------------------------------------------------------------------------------------

\begin{resume}

%----------------------------------------------------------------------------------------
%        CONTACT
%----------------------------------------------------------------------------------------
\section{ Contact\\Information}
Brookhaven National Laboratory \\
Building 741 \hfill (o) 631.344.3146\\
P.O. Box 5000 \hfill {(c)} 914.523.7002 \\
Upton, NY 11973-5000 \hfill {(e)} {\tt tcaswell@\{bnl.gov, gmail.com\}}\\
{\itshape Date of Birth:} March 19, 1985 \hfill {(e)} {\tt tac2205@columbia.edu}\\
{\itshape Citizenship:} USA

%----------------------------------------------------------------------------------------
%	OBJECTIVE SECTION
%----------------------------------------------------------------------------------------

% \section{OBJECTIVE}

% A position in the field of computers with special interests in business applications programming, information processing, and management systems.

%----------------------------------------------------------------------------------------
%	EDUCATION SECTION
%----------------------------------------------------------------------------------------


\section{Appointments}

{\bf Brookhaven National Laboratory }, Upton NY \hfill\\
{\it National Synchrotron Lightsource II (NSLS-II)} \\
Computational Scientist \hfill October 2020 - Present\\
Associate Computational Scientist \hfill January 2017 - October 2020\\
Assistant Computational Scientist \hfill April 2015 - January 2017\\
Research Associate \hfill January 2014 - April 2015

{\bf Columbia University}, New York, NY \hfill\\
{\it Department of Applied Physics and Applied Mathematics}\\
Visiting Associate Research Scientist \hfill October 2019 - Present

\section{Education}

{\bf University of Chicago}, Chicago IL \\
Ph.D., Physics \hfill Spring 2014\\
M.S. Physics \hfill Fall 2008 \\
{\itshape Advisors}: Sidney R. Nagel and Margaret L. Gardel

{\bf Cornell University}, Ithaca, NY \\
B.A., Physics \& Mathematics \hfill Spring 2007\\
{\itshape Research Advisor:}  Sol M. Gruner

\section{Grants}
{\bf Chan Zuckerberg Initiative Essential Open Source Software for Science} \hfill \\
``Matplotlib - Foundation of Scientific Visualization in Python''\\
Thomas A. Caswell (PI) \\
250k\$ \hfill November 2020 - November 2021\\
{\bf Chan Zuckerberg Initiative Essential Open Source Software for Science} \hfill \\
``Matplotlib - Foundation of Scientific Visualization in Python''\\
Thomas A. Caswell (PI) \\
250k\$ \hfill November 2019 - November 2020\\
{\bf BNL Lab Directed Research and Development}\\
``Accelerating materials discovery with total scattering via machine learning''\\
Daniel Olds (PI), Stuart Campbell, Thomas A. Caswell \\
1 post-doc \hfill 2 years



%----------------------------------------------------------------------------------------
%	RESEARCH SECTION
%----------------------------------------------------------------------------------------

%% \section{Outreach}
%% {\bf Physics with a BANG!}\\
%% Annual open house and demo extravaganza.  \hfill 2009-2013\\
%% {\it tour guide, lab representative}

%% {\bf Science club} \hfill 2012-2013 \\
%% Assist in delivering after-school science club at Andrew Carnegie Elementary School

%----------------------------------------------------------------------------------------
%	COMPUTER SKILLS SECTION
%----------------------------------------------------------------------------------------

\section{Open Source\\Contributions}
        {\tt Matplotlib} \hfill 2012-Present
        \begin{itemize}
  \setlength{\itemsep}{-1pt}
        \item {\bf Project Lead} \hfill 2016-Present
        \item{\bf Co-lead Developer}\hfill 2014-2016
        \item {\bf Release Manager} (1.4, 1.5, 2.0, 2.1, 2.2, 3.0 \& 3.2 series)  \hfill 2014-2018, 2020
        \end{itemize}

        {\tt h5py}: {\bf Release Manager} (2.7, 2.8, \& 2.9 series)
        \hfill 2016-2020\\
        {\tt cycler}: {\bf Lead Developer} \hfill 2015-Present\\
        {\tt trackpy}, {\tt pims}: {\bf core developer} \hfill 2013-Present \\
        {\tt pytables}, {\tt numpy}, {\tt scipy}, {\tt conda}, {\tt
          IPython}, {\tt pandas}, \\ {\tt cython}, {\tt cpython}: {\bf contributor} \hfill sporadic



\section{Service}

`Scipy tools plenary session' {\bf co-chair} for SciPy \hfill 2018-2021\\
pydata Global organizing committee  \hfill 2020\\
pydataNYC organizing committee, extracurricular chair \hfill 2019\\
Member of {\tt nteract} Code-of-conduct committee \hfill 2018-Present\\
Recruitment and selection panels for scientific staff at BNL\hfill 2016-Present\\
`Scipy tool stack' {\bf track chair} for SciPy2017 \hfill 2017\\
Member of {\tt Matplotlib} financial committee \hfill 2015-Present\\
Member of {\tt numpy} financial committee \hfill 2015-Present\\
PyGotham Selection Committee, \hfill 2015\\
University of Chicago Graduate Admissions Committee \hfill 2009

\section{Awards}
Brookhaven National Laboratory Lab Award \hfill 2017\\
Google Open Source Peer Bonus \hfill 2017

\section{Grant Review}
Chan Zuckerberg Human Cell Atlas \hfill 2018, 2019

\section{Society Membership}

American Physical Society \hfill 2012-2014, 2021\\
Python Software Foundation (contributing member) \hfill 2018-Present\\
NumFOCUS (contributing member) \hfill 2016-Present

\section{Publications (selected)}

\begin{enumerate}
  \setlength{\itemsep}{-1pt}
  \item Maffettone et al. \textbf{Gaming the beamlines—employing reinforcement
    learning to maximize scientific outcomes at large-scale user facilities},
    Machine Learning: Science and Technology (2021) 2, 2, 025025
  \item Tafliovic et al. \textbf{Teaching Software Engineering with
    Free Open Source Software Development: An Experience Report},
    Proceedings of the 52nd Hawaii International Conference on System
    Sciences (2019)

  \item Thomas A Caswell. \textbf{Dynamics of the vapor layer below a
    Leidenfrost drop}, PRE 90, 013014 (2014)

  \item Thomas A Caswell et al. \textbf{Observation and
    Characterization of the Vestige of the Jamming Transition in a
    Thermal 3D System}, PRE 87, 012303 (2013)
    % , Zexin Zhang, Margaret L Gardel, and Sidney R Nagel

  \item Thomas A Caswell et al. \textbf{A High Speed Area
    Detector for Novel Imaging Techniques in a Scanning Transmission
    Electron Microscope} Ultramicroscopy 109, 304-311 (2009)
    %, Peter Ercius, Mark W Tate, Alper  Ercan, Sol M Gruner, David A Muller.

\end{enumerate}

\section{Invited Talks (selected)}
\begin{enumerate}
    \setlength{\itemsep}{-1pt}
\item Thomas A. Caswell \textbf{Matplotlib and Scientific Visualization},
  APS March Meeting (2021-03)
\item Thomas A. Caswell, \textbf{h5py latest developments}, HDF5
  European Workshop for Science and Industry (2019-09)
\item Thomas A. Caswell, \textbf{Matplotlib 2.0: One does not simply
  change the defaults}, PLOTCON NYC (2016-11)
\item Thomas A. Caswell, \textbf{Time series exploration with
  Matplotlib}, PyData DC (2016-10)

\end{enumerate}




\section{Contributed Talks (selected)}

\begin{enumerate}
    \setlength{\itemsep}{-1pt}
\item Thomas A Caswell \textbf{Separations of \sout{concerns} scales} PyData
  Global (2020-11)
\item Hannah Aizenman and Thomas A. Caswell \textbf{Introduction to
  Matplotlib} SciPy (2019-07)
\item Thomas A. Caswell, \textbf{Interactive Matplotlib Tutorial}
  pydata NYC (2017-11)
\end{enumerate}



% \newpage

% \section{Motivation}

% I enjoy building tools, both hardware and software, that enable
% research that could not be done without those tools.  I have a strong
% interest in imaging of any modality (optical, x-ray, STEM, CT, etc.)
% and image analysis, in particular acquiring, processing, and
% visualizing large multi-dimensional data. I am looking for a post-doc
% position where I can push the limits of either instrumentation or
% analysis in the context of condensed matter physics.

% I am aiming to defend my Ph.D. in December 2013 and have the long-term
% goal of a permanent staff scientist position at a research institute,
% national lab, or other user facility.

\end{resume}



\end{document}
